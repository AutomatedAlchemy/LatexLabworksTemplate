% =============================================================================
% ==     ALLGEMEINE VORLAGE FÜR WISSENSCHAFTLICHE PROTOKOLLE                 ==
% =============================================================================

\documentclass[a4paper,12pt,ngerman]{article}

% =============================================================================
% == METADATEN DES DOKUMENTS (HIER FÜR NEUEN VERSUCH ANPASSEN)               ==
% =============================================================================

% --- Allgemeine Informationen ---
\newcommand{\Titel}{[Titel des Versuchs]}
\newcommand{\Untertitel}{Versuchsprotokoll}
\newcommand{\Kurs}{[Name des Kurses oder Praktikums]}
\newcommand{\Studiengang}{[Ihr Studiengang]}
\newcommand{\Hochschule}{[Name der Hochschule und Fakultät]}

% --- Autoren ---
\newcommand{\Autoren}{[Name 1], [Name 2], [Name 3]}

% --- Termine und Betreuung ---
\newcommand{\Versuchsdatum}{TT. Monat JJJJ}
\newcommand{\Abgabedatum}{TT. Monat JJJJ}
\newcommand{\Betreuer}{[Name des Betreuers]}


% =============================================================================
% == PAKETE UND EINSTELLUNGEN (Normalerweise keine Änderung nötig)           ==
% =============================================================================

% -----------------------------------------------------------------------------
% -- Sprach- und Zeichenkodierung
% -----------------------------------------------------------------------------
\usepackage[utf8]{inputenc}
\usepackage[T1]{fontenc}
\usepackage[ngerman]{babel}
\usepackage{csquotes}

% -----------------------------------------------------------------------------
% -- Schriftart und Seitenlayout
% -----------------------------------------------------------------------------
\usepackage{lmodern}
\usepackage[a4paper, top=2.5cm, bottom=2.5cm, left=2.5cm, right=2.5cm]{geometry}
\usepackage{setspace}
\onehalfspacing
\setlength{\parindent}{0pt}
\setlength{\parskip}{1.5ex}

% -----------------------------------------------------------------------------
% -- Typografie
% -----------------------------------------------------------------------------
\usepackage{microtype}

% -----------------------------------------------------------------------------
% -- Tabellen
% -----------------------------------------------------------------------------
\usepackage{booktabs}
\usepackage{tabularx}
\usepackage{placeins}
\usepackage{makecell}
\renewcommand\cellalign{tl}
\renewcommand{\arraystretch}{1.3}

% -----------------------------------------------------------------------------
% -- Mathematik, Physik, Chemie und Einheiten
% -----------------------------------------------------------------------------
\usepackage{amsmath}
\usepackage{physics}
\usepackage[version=4]{mhchem}
\usepackage{siunitx}
\sisetup{
  locale = DE,
  detect-all,
  round-mode = places,
  round-precision = 2
}

% -----------------------------------------------------------------------------
% -- Abbildungen und Gleitobjekte
% -----------------------------------------------------------------------------
\usepackage{graphicx}
\usepackage{float}
\usepackage{caption}
\usepackage{subcaption}
\captionsetup{font=small, labelfont=bf, singlelinecheck=false, format=hang}
\addto\captionsngerman{\renewcommand{\figurename}{Abb.}}
\addto\captionsngerman{\renewcommand{\tablename}{Tab.}}
\graphicspath{{Abbildungen/}}

% -----------------------------------------------------------------------------
% -- Hyperlinks und PDF-Erstellung
% -----------------------------------------------------------------------------
\usepackage{hyperref}
\hypersetup{
  pdftitle={\Titel},
  pdfauthor={\Autoren},
  pdfsubject={\Untertitel},
  pdfcreator={LatexLabworksTemplate - github.com/Probst1nator/LatexLabworksTemplate},
  colorlinks=true,
  linkcolor=black,
  citecolor=blue,
  urlcolor=blue,
  breaklinks=true,
  hidelinks
}

% -----------------------------------------------------------------------------
% -- Abkürzungsverzeichnis
% -----------------------------------------------------------------------------
\usepackage{acro}
%
% Fügen Sie hier Ihre eigenen Akronyme nach dem folgenden Schema hinzu:
% \DeclareAcronym{<label>}{ short = <Kurzform>, long = <Langform> }
% Beispiel:
% \DeclareAcronym{fau}{ short = FAU, long = Friedrich-Alexander-Universität }
%

% -----------------------------------------------------------------------------
% -- Literaturverzeichnis mit BibLaTeX
% -----------------------------------------------------------------------------
\usepackage[
  backend=biber,
  style=numeric,
  sorting=none,
  natbib=true
]{biblatex}
\addbibresource{quellen.bib} % Name der .bib-Datei


% =============================================================================
% ==                          DOKUMENTBEGINN                                 ==
% =============================================================================
\begin{document}

% -----------------------------------------------------------------------------
% -- Deckblatt
% -----------------------------------------------------------------------------
\begin{titlepage}
  \begin{flushleft}
    % Fügen Sie hier bei Bedarf ein Logo ein, z.B.:
    % \includegraphics[width=8cm]{logo.png}
  \end{flushleft}

  \vspace{2cm}

  \begin{center}
    {\large\bfseries \Kurs\par}
    \vspace{1.5cm}
    {\huge\bfseries \Titel\par}
    \vspace{2cm}
    {\Large\bfseries \Untertitel\par}

    \vspace{1cm}

    {\normalsize
    An der\\
    \Hochschule
    }
    
    \vspace{1cm}
    
    {\normalsize \Studiengang\\}

    \vspace{2cm}

    {\normalsize vorgelegt von\\
    \vspace{0.5cm}
    {\large \Autoren}\\
    }

    \vfill % Schiebt den unteren Teil an das Seitenende

    {\normalsize
    Versuchsdurchführung: \Versuchsdatum\\
    Abgabedatum: \Abgabedatum
    }
    
    \vspace{1cm}
    
    {\normalsize Versuchsbetreuer: \Betreuer}
  \end{center}
\end{titlepage}
\pagenumbering{gobble}

% -----------------------------------------------------------------------------
% -- Verzeichnisse
% -----------------------------------------------------------------------------
\newpage
\cleardoublepage
\pagenumbering{Roman}
\tableofcontents
\thispagestyle{plain}

\newpage
\listoffigures
\addcontentsline{toc}{section}{Abbildungsverzeichnis}
\vspace{2em}
\listoftables
\addcontentsline{toc}{section}{Tabellenverzeichnis}

\newpage
\printacronyms[name=Abkürzungsverzeichnis]
\addcontentsline{toc}{section}{Abkürzungsverzeichnis}
\clearpage
\pagenumbering{arabic}

% =============================================================================
% ==                         HAUPTTEIL DES PROTOKOLLS                        ==
% =============================================================================
\section{Einleitung}
%
% Fügen Sie hier Ihren Einleitungstext ein.
% Leiten Sie das Thema des Versuchs her.
% Erklären Sie die Motivation und die Relevanz des Themas.
% Formulieren Sie die genaue Zielsetzung bzw. die Forschungsfrage des Versuchs.
%

\clearpage
\section{Grundlagen}
%
% Fügen Sie hier die theoretischen Grundlagen ein.
% Erläutern Sie die physikalischen, chemischen oder technischen Prinzipien,
% die für das Verständnis des Versuchs notwendig sind.
% Leiten Sie relevante Formeln her und erklären Sie die darin vorkommenden Größen.
% Jede nicht-triviale Aussage sollte mit einer Quelle belegt werden, z.B. \cite{IhreQuelle}.
%

\clearpage
\section{Durchführung}
%
% Beschreiben Sie hier den Versuchsaufbau und den Ablauf.
% Schreiben Sie im Präteritum und in einer sachlichen, unpersönlichen Form.
% Nennen Sie alle verwendeten Geräte, Materialien und Chemikalien.
% Beschreiben Sie die durchgeführten Schritte so präzise, dass der Versuch
% von einer anderen Person reproduziert werden könnte.
%

\clearpage
\section{Ergebnisse und Diskussion}
%
% Präsentieren und diskutieren Sie hier Ihre Messergebnisse.
%
% 1. Präsentation der Ergebnisse:
%    - Stellen Sie Ihre Daten übersichtlich in Form von Diagrammen, Abbildungen
%      oder Tabellen dar.
%    - Jede Abbildung und Tabelle benötigt eine nummerierte Beschriftung.
%    - Beschreiben Sie objektiv, was in den Daten zu sehen ist.
%
% 2. Diskussion der Ergebnisse:
%    - Interpretieren Sie Ihre Beobachtungen.
%    - Setzen Sie die Ergebnisse in den Kontext der theoretischen Grundlagen.
%    - Vergleichen Sie Ihre Ergebnisse mit Literaturwerten, falls vorhanden.
%    - Diskutieren Sie mögliche Fehlerquellen, Messunsicherheiten und deren
%      Einfluss auf die Ergebnisse.
%

\clearpage
\section{Zusammenfassung}
%
% Fassen Sie die wichtigsten Ergebnisse und Schlussfolgerungen des Versuchs
% kurz und prägnant zusammen.
% Beantworten Sie die in der Einleitung formulierte Zielsetzung.
% Geben Sie gegebenenfalls einen kurzen Ausblick.
%

% =============================================================================
% ==                ABSCHLUSS: QUELLENVERZEICHNIS & ANHANG                   ==
% =============================================================================
\clearpage
\printbibliography[heading=bibintoc, title={Quellenverzeichnis}]

\clearpage
\appendix
\section{Anhang}
\label{sec:Anhang}
%
% Fügen Sie hier bei Bedarf zusätzliches Material ein, z.B.:
% - Rohdatentabellen
% - Zusätzliche Diagramme oder Abbildungen
% - Detaillierte Berechnungen
% - Sicherheitshinweise
%

\end{document}
