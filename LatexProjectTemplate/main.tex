% =============================================================================
% ==     ALLGEMEINE VORLAGE FÜR WISSENSCHAFTLICHE PROTOKOLLE                 ==
% =============================================================================

% OPTIMIERUNG: KOMA-Script (scrartcl) statt article
% Bessere Typografie für europäische Dokumente und A4
\documentclass[
  paper=a4,
  fontsize=12pt,
  ngerman,
  parskip=half,      % Erzeugt sauberen Abstand zwischen Absätzen (kein Einrücken)
  listof=totoc,      % Abbildungs-/Tabellenverzeichnis im Inhaltsverzeichnis
  bibliography=totoc % Literaturverzeichnis im Inhaltsverzeichnis
]{scrartcl}

% =============================================================================
% == METADATEN (HIER ANPASSEN)                                               ==
% =============================================================================

\newcommand{\Titel}{[Titel des Versuchs]}
\newcommand{\Untertitel}{Versuchsprotokoll}
\newcommand{\Kurs}{[Name des Kurses oder Praktikums]}
\newcommand{\Studiengang}{[Ihr Studiengang]}
\newcommand{\Hochschule}{[Name der Hochschule]}
\newcommand{\Autoren}{[Name 1], [Name 2]}
\newcommand{\Versuchsdatum}{TT. Monat JJJJ}
\newcommand{\Abgabedatum}{TT. Monat JJJJ}
\newcommand{\Betreuer}{[Name des Betreuers]}

% =============================================================================
% == PAKETE UND EINSTELLUNGEN                                                ==
% =============================================================================

% --- Sprache & Encoding ---
\usepackage[utf8]{inputenc}
\usepackage[T1]{fontenc}
\usepackage[ngerman]{babel}
\usepackage{csquotes}

% --- Schriftart & Layout ---
\usepackage{lmodern}
\usepackage[top=2.5cm, bottom=2.5cm, left=2.5cm, right=2.5cm]{geometry}
\usepackage[onehalfspacing]{setspace} % Zeilenabstand 1.5

% --- Typografie ---
\usepackage{microtype} % Verhindert "löchrigen" Textblocksatz

% --- Tabellen ---
\usepackage{booktabs} % Für professionelle Linien (toprule, midrule...)
\usepackage{tabularx}
\usepackage{placeins} % \FloatBarrier Befehl
\renewcommand{\arraystretch}{1.2} % Tabellen etwas luftiger gestalten

% --- Mathe & Naturwissenschaften ---
\usepackage{amsmath}
\usepackage{physics}
\usepackage[version=4]{mhchem}
\usepackage{siunitx}
\sisetup{
  locale = DE,
  detect-all,
  per-mode = symbol,
  uncertainty-mode = separate % Ausgabe: (1.23 +/- 0.05) statt 1.23(5)
}

% --- Bilder & Grafiken ---
\usepackage{graphicx}
\usepackage{float}
\usepackage[format=plain, labelfont={bf,sf}, textfont=small]{caption}
\usepackage{subcaption}
\graphicspath{{Abbildungen/}}

% --- Quellcode (Listings) ---
\usepackage{listings}
\usepackage{xcolor}
\definecolor{codegray}{rgb}{0.5,0.5,0.5}
\definecolor{codeblue}{rgb}{0.0,0.0,0.6}
\lstset{
    basicstyle=\ttfamily\small,
    commentstyle=\color{codegray},
    keywordstyle=\color{codeblue},
    numbers=left,
    numberstyle=\tiny\color{codegray},
    breaklines=true,
    frame=single,
    captionpos=b,
    keepspaces=true
}

% --- Todos (Hilfreich beim Schreiben) ---
% Nutze \todo{Notiz} im Text für Randnotizen.
% Vor der Abgabe: Setze die Option 'disable' in die eckigen Klammern unten.
\usepackage[colorinlistoftodos, prependcaption, textsize=tiny]{todonotes}

% --- Referenzen & Links ---
\usepackage[
  pdftitle={\Titel},
  pdfauthor={\Autoren},
  hidelinks,       % Keine roten Rahmen um Links
  colorlinks=true, % Text farbig statt Rahmen
  linkcolor=black, % Inhaltsverzeichnis schwarz lassen
  citecolor=darkgray,
  urlcolor=blue
]{hyperref}

% --- Intelligente Referenzen (Cleveref) ---
% Immer NACH hyperref laden!
% Erlaubt \cref{fig:bild} -> Ausgabe "Abb. 1" automatisch
\usepackage[ngerman]{cleveref}

% --- Abkürzungen ---
\usepackage{acro}
% Beispiel: \DeclareAcronym{led}{short=LED, long=Light Emitting Diode}

% --- Literatur ---
\usepackage[backend=biber, style=numeric, sorting=none]{biblatex}
\addbibresource{quellen.bib}


% =============================================================================
% == DOKUMENTBEGINN                                                          ==
% =============================================================================
\begin{document}

% --- Deckblatt ---
\begin{titlepage}
    \centering
    \vspace*{1cm}
    {\scshape\LARGE \Hochschule \par}
    \vspace{1cm}
    {\scshape\Large \Kurs \par}
    \vspace{1.5cm}
    {\huge\bfseries \Titel \par}
    \vspace{0.5cm}
    {\Large\bfseries \Untertitel \par}
    
    \vspace{2cm}
    
    {\Large \Autoren \par}
    
    \vfill
    
    % Tabelle für saubere Ausrichtung der Metadaten
    \begin{tabular}{rl}
        \textbf{Betreuer:} & \Betreuer \\
        \textbf{Durchführung:} & \Versuchsdatum \\
        \textbf{Abgabe:} & \Abgabedatum
    \end{tabular}
    \vspace{1cm}
\end{titlepage}

% --- Verzeichnisse ---
\pagenumbering{Roman}

\tableofcontents
\listoffigures
\listoftables
\printacronyms[name=Abkürzungsverzeichnis]

\clearpage
\pagenumbering{arabic}

% =============================================================================
% == INHALT                                                                  ==
% =============================================================================

\section{Einleitung}
%
% Füge hier deinen Einleitungstext ein.
% 1. Leite das Thema des Versuchs her.
% 2. Erkläre die Motivation und die Relevanz des Themas.
% 3. Formuliere die genaue Zielsetzung bzw. die Forschungsfrage.
%
% TIPP: Nutze \todo{Hier fehlt noch Text} um dir Notizen im PDF zu machen.
%

\clearpage
\section{Grundlagen}
%
% Füge hier die theoretischen Grundlagen ein.
% - Erläutere die physikalischen/chemischen/technischen Prinzipien.
% - Leite relevante Formeln her und erkläre die Größen.
% - Belege Aussagen mit Quellen, z.B. \cite{DeineQuelle}.
%
% TIPP ZU REFERENZEN:
% Nutze den Befehl \cref{labelname} statt "Abbildung \ref{...}".
% Beispiel: "Wie in \cref{eq:einstein} zu sehen ist..."
%
\begin{equation}
    E = mc^2
    \label{eq:einstein}
\end{equation}

\clearpage
\section{Durchführung}
%
% Beschreibe hier den Versuchsaufbau und den Ablauf.
% - Schreibe im Präteritum und unpersönlich ("Wurde gemessen" statt "Ich habe gemessen").
% - Nenne verwendete Geräte und Materialien.
% - Falls du Code verwendet hast, nutze die lstlisting-Umgebung (siehe unten).
%

% Beispiel für Code-Einbindung:
% \begin{lstlisting}[language=Python, caption={Messscript}, label={code:py}]
% def start():
%     return True
% \end{lstlisting}

\clearpage
\section{Ergebnisse und Diskussion}
%
% Präsentiere und diskutiere hier deine Messergebnisse.
%
% 1. Präsentation:
%    - Diagramme, Abbildungen oder Tabellen.
%    - Jede Abbildung braucht eine Beschriftung (caption) und ein Label.
%    - Beschreibe objektiv, was zu sehen ist.
%
% 2. Diskussion:
%    - Interpretiere deine Beobachtungen.
%    - Vergleiche mit Literaturwerten.
%    - Diskutiere Fehlerquellen und Unsicherheiten (Fehlerrechnung!).
%

% Beispielplatzhalter für ein Bild:
% \begin{figure}[H]
%     \centering
%     \includegraphics[width=0.8\textwidth]{DeinBild.png}
%     \caption{Beschriftung des Bildes}
%     \label{fig:messung1}
% \end{figure}

\clearpage
\section{Zusammenfassung}
%
% Fasse die wichtigsten Ergebnisse kurz zusammen.
% - Wurde das Ziel erreicht?
% - Gib einen kurzen Ausblick.
%

% =============================================================================
% == ANHANG & QUELLEN                                                        ==
% =============================================================================
\clearpage
\printbibliography

\clearpage
\appendix
\section{Anhang}
\label{sec:Anhang}
%
% Hier Platz für Rohdaten, zusätzliche Plots oder Herleitungen.
%

\end{document}